Un péndulo doble, también conocido como péndulo caótico, es un péndulo con otro péndulo unido a su extremo, 
formando un sistema físico simple que exhibe un comportamiento dinámico rico con una fuerte sensibilidad a las condiciones iniciales.
Consideremos las siguientes variables:
g = 9.81; Aceleración de la gravedad
m_1: Masa del primer pendulo
m_2: Masa del segundo pendulo
theta_1: Angulo del primer péndulo
theta_2: Angulo del segundo péndulo
L_1: Longitud del primer péndulo
L_2: Longitud del segundo péndulo
La variación del angulo viene dada por la ecuación diferencial:
$\ddot{\theta_1}=\frac{-g(2_m1+m_2)sin(\theta_1)-m_2gsin(\theta_1-2\theta_2)}$
** Terminar de escribir la ecuación diferencial https://www.myphysicslab.com/pendulum/double-pendulum-en.html
Podemos resolver la ecuación diferencial con el método de Runge-Kutta de cuarto orden.